In most cases, the coarse design of my backpacks is a basic rectangular prism. The dimensions however, are not that easy to figure out, and depend on the target volume as well as how I wear a pack, my torso length and my shoulder width. I also consider aesthetics as a factor, and try wherever possible to stick to pleasing forms and harmonious dimensions. For the latter, I often refer to the \textit{golden ratio} which has been used across the ages for bringing mathematical harmony to art, architecture, and many other applications.

\begin{note}
  \index{divine proportion}
  The golden ratio (also known as the divine proportion) is an irrational number with deep mathematical roots (see Fibonacci sequence for more). In its most simple use, it can offer a eye-pleasing ratio between the length $a$ and the width $b$ of a rectangle ($a > b > 0$).

  \begin{equation}
    \centering
    \frac{a + b}{a} = \frac{a}{b}=\varphi
  \end{equation}

  And through clever manipulation and substitution, you achieve an exact formulae:

  \begin{equation}
    \centering
    \label{eq:golden-ratio}
    \varphi^2 - \varphi - 1 = 0
  \end{equation}

  The golden ratio $\varphi$ should solve the equation \ref{eq:golden-ratio}, and its only positive value is as follow:

  \begin{equation}
    \centering
    \varphi = \frac{1 + \sqrt{5}}{2} = 1.6180339\dots
  \end{equation}

\end{note}

\subsection{How to read the sketches (hidden seams)}

In most of the upcoming drawings I only highlight the basic dimensions of the finished panels and the outer seam allowance.

\begin{warning}
  The different panel's construction seams and extra fabric should be accounted for before cutting the panels.
\end{warning}

Before cutting any panels for a backpack, have a thorough look at section \ref{sec:construction-details} where you should pay attention to all the seams and extra fabric necessary to bind and build all the features of each panel.

The illustration \ref{ske:panel-back-seam-highlight} is an example of how seams can impact fabric cut-out of the panel shown in illustration \ref{ske:panel-back-design}, and why you should always have a few drawing of the features of your pack. In this instance, I would cut 4 separate panels and assemble them when working on each feature.

\begin{note}
  When working on design which involve multiple parts per panel, you also want to consider the fabric's patterns, and make sure they match!
\end{note}

\illustrate[.75]{media/sketches/panel-back-seam-highlight.pdf}
{Be careful to always consider your dimensions with and without the extra seams}
{ske:panel-back-seam-highlight}

\subsection{The basic shape}

As I just expressed my interest in balancing the shapes and dimensions of the bag through the golden ratio, I should start by saying this is not an exact science. I mostly use it to draw the basic shapes and arrange the different features together, and then, I often deviate a little. But all in all, it's a guideline.

\illustrate{media/sketches/pack-full-3d.pdf}
{Crude 3D sketch of backpack shape and size}
{ske:pack-full-3d}

\illustrate{media/sketches/pack-rough-cut.pdf}
{Rough backpack dimensions including seam allowance of 10mm}
{ske:pack-rough-cut}

\index{golden ratio}
In the illustration \ref{ske:pack-rough-cut}, the golden ratio is not obvious, but it is there. If we look at the front panel, the golden ratio is only supposed to impact the assembled pack, and therefore not visible in the fabric cut-outs ($\frac{750}{300} \gg 1.6$), but if you account for the $20cm$ planned for the roll-top and include some compression, you're getting pretty close ($\frac{550}{300} = 1.8$). I actually do prefer a slimmer backpack, but one could definitely get closer to the $1.6$ in any design.

I have tried many different combinations and ratios, and I find the best looks comes from ratios between $1.65 < \varphi_{front} < 1.85$. These also account to some extent to my torso length (as I am slim and tall).

\begin{note}
  \index{roll-top}
  The 20cm reserved to the roll-top is not for waterproofness or aesthetics, but about the extra volume this offers (approx 6 litres) if I need to carry more bulk.
\end{note}

\index{shoulder width}
With that in mind, my second constraint is given by the overall pack width I am aiming for, and that depends mostly on my shoulder width. There is a very simple reason for that: it's not actually about the shoulder width itself (that impacts more the straps design) but it is more about the freedom of movement of my arms. I do not want my backpack to to be wider than my rib cage, and for my arms and elbows to constantly hit and rub on the pack and create discomfort.

\subsection{The front panel: aesthetics \& usability}

\illustrate[.75]{media/sketches/panel-front-design.pdf}
{Concept of the front panel designed for usability}
{ske:panel-front-design}

\subsection{The side panels: compression}

One proposal for the straps spacing would be (like in this drawing) $A=200mm$, $B=110mm$ and $C=85mm$.

\illustrate[.55]{media/sketches/panel-side-design.pdf}
{Concept of the side panels designed for pack compression}
{ske:panel-side-design}

\subsection{The bottom panel: robustness}
\subsection{The back panel: comfort}

\illustrate[.75]{media/sketches/panel-back-design.pdf}
{Concept of the back panel designed for comfort}
{ske:panel-back-design}

\subsection{The shoulder straps: lightweight}
\subsection{The hip belt: padding}
