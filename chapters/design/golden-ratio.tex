When I design a backpack, one of the major issue I face is getting the dimensions right, and how features are placed in relation to one another. From choosing the dimensions of the pack itself to settling on where compression straps are placed on the side panels or making up my mind on the size and placement of some mesh pockets, making an informed choice can be quite cumbersome.

Now, there is really only two criteria to consider: usability and eye-candy. Let's face it, when I spend 30 hours building a pack - which I do more often than not - I want the end result to be functional, but also pleasing enough so I can wear it proudly in public.

\begin{quote}
  When in doubt: golden ratio!
\end{quote}

Across the ages, when working out forms and dimensions in relations to each other, the golden ratio has played a major part of many concepts, from the roman architecture and the renaissance painting to many natural shapes and forms.

You can find a quick mathematical explanation below of how this number came to life, but for reference, the golden ratio is defined as follow.

\begin{equation}
  \varphi \approx 1.62
\end{equation}

The many applications of $\varphi$ revolve around the aesthetics of geometrical forms. A concrete list of example applications could be:

\begin{itemize}
  \item Define the basic rectangular shape of the backpack
  \item How high should a side pocket go
  \item Where to put compression straps
  \item How short a zipper closure can be
  \item Where to place load lifters
  \item How to offset a chest strap's or hip belt's quick release buckle
\end{itemize}

Now, I don't always rely on this ratio when working on a design, but it's a general guideline for these times where functionality and prior experience did not provide me with an answer. Have a look at the chapter \ref{chap:dimensions} to see some more concrete examples of where I use this ratio.

\begin{note}
  \index{divine proportion}
  The golden ratio (also known as the divine proportion) is an irrational number with deep mathematical roots (see Fibonacci sequence for more). In its most simple use, it can offer a eye-pleasing ratio between the length $a$ and the width $b$ of a rectangle ($a > b > 0$).

  \begin{equation}
    \centering
    \frac{a + b}{a} = \frac{a}{b}=\varphi
  \end{equation}

  And through clever manipulation and substitution, you achieve an exact formulae:

  \begin{equation}
    \centering
    \label{eq:golden-ratio}
    \varphi^2 - \varphi - 1 = 0
  \end{equation}

  The golden ratio $\varphi$ should solve the equation \ref{eq:golden-ratio}, and its only positive value is as follow:

  \begin{equation}
    \centering
    \varphi = \frac{1 + \sqrt{5}}{2} = 1.6180339\dots
  \end{equation}
\end{note}
