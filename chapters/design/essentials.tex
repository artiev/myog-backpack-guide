There is a backpack for any situation, and it's always interesting to look at what others are doing. I have a few backpacks - both commercial and custom made - at home I can analyse inside and out to figure out what I'm interested in. And there is a plethora of detailed pictures and designs on the web, ranging from home made MYOG articles, to high-end ultralight gear manufacturer, to cost-effective mass production design.

\subsection{Features}

When I try to define a set of features for a new backpack, the first step is to define the style and application of my design:

\begin{description}

  \item [Aesthetics] As basic as this sounds, I always enjoy a pack that doesn't look like I put it together in a few minutes. This will cost some extra fabric and or components in most cases, but I prefer it that way.

  \index{padding}
  \item [Load] The padding, weight balance and distribution, boiling down to the basic shape of a backpack is driven by how much weight it will need to carry. This impacts the construction of the shoulder straps, and the of the hip belt as well.

  \item [Weight] The basic weight of the backpack is tied to its construction, the materials used and its shape, the features I build in, etc... I rarely use this as a target, rather, I review everything else and try hard not to over-engineer a pack to limit the extra weight.

  \index{attachment}
  \item [Attachments] There is a ton of options to attach things to a backpack. In outdoor cases, I often have a tent and/or sleeping pad attached on the sides and bottom of the pack, I also use walking poles which I often strap on the side of the pack.

  \index{pocket}
  \item [Pockets] Whether inside or outside, pockets are an essential part of a pack. Depending on how you usually organise your pack, you would consider internal compartments, inside pockets, outside mesh pockets, and so on. I personally don't care much for pockets, as I organise my gear in stuff sacks. But I often need to attach/store things on the outside for easy access, or for drying.

  \index{roll-top}
  \index{velcro}
  \item [Closure] There is a ton of different way to close a backpack. To name but a few, I often consider roll tops, zippers, velcro, tie-ins or a mix of these.

  \index{padding}
  \item [Padding] Padding can be essential, or completely superfluous depending on what you carry, and how you carry it. I always build in a sleeve of some kind following the entire width and length of the pack to be able to add/remove foam padding depending on the trip.

  \index{hip-belt}
  \item [Hip-Belt] For me, and essential part of any hiking/trekking pack. The rare packs I built without a hip-belt, I gave away because I never use them. Now whether I want the hip-belt complete with padding, or just a simple strap depends on the weight I plan on carrying. But it's always a good weight investment.

\end{description}
