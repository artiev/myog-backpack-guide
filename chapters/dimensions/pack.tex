\begin{warning}
  The different panel's construction seams and extra fabric should be accounted for before cutting the panels. Refer to section \ref{sec:seams-and-cutouts} for more.
\end{warning}

As I just expressed my interest in balancing the shapes and dimensions of the bag through the golden ratio, I should start by saying this is not an exact science. I mostly use it to draw the basic shapes and arrange the different features together, and then, I often deviate a little. But all in all, it's a guideline.

\illustrate{media/sketches/pack-full-3d.pdf}
{Crude 3D sketch of backpack shape and size}
{ske:pack-full-3d}

\illustrate{media/sketches/pack-rough-cut.pdf}
{Rough backpack dimensions including seam allowance of 10mm}
{ske:pack-rough-cut}

\index{golden ratio}
In the illustration \ref{ske:pack-rough-cut}, the golden ratio is not obvious, but it is there. If we look at the front panel, the golden ratio is only supposed to impact the assembled pack, and therefore not visible in the fabric cut-outs ($\frac{750}{300} \gg 1.6$), but if you account for the $20cm$ planned for the roll-top and include some compression, you're getting pretty close ($\frac{550}{300} = 1.8$). I actually do prefer a slimmer backpack, but one could definitely get closer to the $1.6$ in any design.

I have tried many different combinations and ratios, and I find the best looks comes from ratios between $1.65 < \varphi_{front} < 1.85$. These also account to some extent to my torso length (as I am slim and tall).

\begin{note}
  \index{roll-top}
  The 20cm reserved to the roll-top is not for waterproofness or aesthetics, but about the extra volume this offers (approx 6 litres) if I need to carry more bulk.
\end{note}

\index{shoulder width}
With that in mind, my second constraint is given by the overall pack width I am aiming for, and that depends mostly on my shoulder width. There is a very simple reason for that: it's not actually about the shoulder width itself (that impacts more the straps design) but it is more about the freedom of movement of my arms. I do not want my backpack to to be wider than my rib cage, and for my arms and elbows to constantly hit and rub on the pack and create discomfort.
