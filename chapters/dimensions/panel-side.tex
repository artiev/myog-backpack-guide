When designing the side panels (both sides are using the same design), my main focus is to provide proper compression straps to help bring the load of the pack - especially its centre of gravity - as close to my back as possible. On top of that, I want to make sure the bulk of the pack and the heavy items can be placed and compressed at the bottom of the pack in order to carry most of the weight on my hips.

To solve this, I always design a side panel with 3 lines of compression, equally spaced out, and overall offset to the upper part of the backpack's main body. In this specific instance, I use 3 separate straps, but for ultralight purposes, you could use a single - static or shock - cord zigzagging through 6 tie-in points (including one adjustable anchor).

Once again, the golden ratio can help you out. If you take the middle strap as the reference, you can place it at $H / \varphi$ where $H$ is the closed pack height. Here, the pack should be roughly 550mm high when closed up, meaning the middle strap can be placed around $550 / 1.62 = 339mm$ from the bottom.

To make things simple based on the illustration \ref{ske:panel-side-design}, you can work your dimensions from:

\begin{equation}
  \left\{
    \begin{array}{rl}
     \varphi &\approx \dfrac{A+B}{B+C}\\
      H &= A+2 \times B+C + 3 \times w_{webbing}\\
    \end{array}
  \right.
\end{equation}

And we already know:

\begin{equation}
  \left\{
    \begin{array}{ll}
     \varphi = 1.62\\
      H = 550mm\\
      w_{webbing} = 15mm
    \end{array}
  \right.
\end{equation}

Now, I'll spare you the math, and just get to the point (I made some approximation to have nice round numbers):

\begin{equation}
  \left\{
    \begin{array}{rl}
      A &= 200mm\\
      B &= 110mm\\
      C &= 85mm
    \end{array}
  \right.
\end{equation}

\illustrate[.55]{media/sketches/panel-side-design.pdf}
{Concept of the side panels designed for pack compression}
{ske:panel-side-design}

Aside from shifting the pack's centre of gravity as low and close to the body as possible, leaving the bottom part of the panel "free" of compression straps also offers the possibility to add side pockets to your pack.

Since I rarely use them on multi-day hikes, I decided against them for this backpack, but for day packs these pockets can come very handy, and I build side pockets in all the time.

\begin{note}
   With this system, you could have a $170 \times 150mm$ side-pocket on each side and use the lowest compression strap as a tightener for, for example, securing a water bottle.
\end{note}

One thing to consider is whether you want the straps to have a quick-release mechanism or if you prefer simple constructions like a ladder-lock. This can have an impact on the usefulness of your straps beyond compression. I always design at least one compression strap with a quick-release buckle, or some equivalent system, so I can effortlessly secure anything to the side of the pack, from a tent, to trekking poles, to a rain jacket. More details on construction options and techniques can be found in section \ref{sec:construction-side-pockets}.
