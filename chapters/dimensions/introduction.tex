In most cases, the coarse design of my backpacks is a basic rectangular prism. The dimensions however, are not that easy to figure out, and depend on the target volume as well as how I wear a pack, my torso length and my shoulder width. I also consider aesthetics as a factor, and try wherever possible to stick to pleasing forms and harmonious dimensions. For the latter, I often refer to the \textit{golden ratio} which has been used across the ages for bringing mathematical harmony to art, architecture, and many other applications.

\begin{note}
  \index{divine proportion}
  The golden ratio (also known as the divine proportion) is an irrational number with deep mathematical roots (see Fibonacci sequence for more). In its most simple use, it can offer a eye-pleasing ratio between the length $a$ and the width $b$ of a rectangle ($a > b > 0$).

  \begin{equation}
    \centering
    \frac{a + b}{a} = \frac{a}{b}=\varphi
  \end{equation}

  And through clever manipulation and substitution, you achieve an exact formulae:

  \begin{equation}
    \centering
    \label{eq:golden-ratio}
    \varphi^2 - \varphi - 1 = 0
  \end{equation}

  The golden ratio $\varphi$ should solve the equation \ref{eq:golden-ratio}, and its only positive value is as follow:

  \begin{equation}
    \centering
    \varphi = \frac{1 + \sqrt{5}}{2} = 1.6180339\dots
  \end{equation}
\end{note}
