There are too many manufacturers of backpacks for me to keep looking at everyone one of them. But over time, and in the lightweight to ultra-lightweight community, there are a couple of well respected names I always look forward to dissecting and use for reference. Smaller names tend to innovate more, where better established brands have more diversity and complexity. Both are a fantastic source for inspiration.

Generally speaking, UL packs have a lot in common, so I don’t focus all my attention on ultralight gear maker, instead, I do all I can to find innovative ways of dialling down on pack weight without compromising comfort too much.

Here is a very short list of where I look for inspiration every once in a while:

\begin{description}

  \item [Reddit] Oh yeah, that’s where you want to go to get in touch with the community of fantastic trekker/hikers who make packs. There are plenty of platform and communities out there, I personally roam these the most: \textit{r/ultralight} and \textit{r/myog}.

  \item [HMG] Hyperlight Mountain Gear is well respected business for a variety of ultralight gear, and I just love their pack designs. I unfortunately do not own one (yet) as I find them a little pricey. But who knows.

  \item [KS Kinpu] San ultralight gear is a fantastic one-man custom pack making business held by Laurent, out of Japan. I personally own his Tao pack, and I’ve had very comfy, very long days of hiking with it. This was the pack I brought with me to a short trek in Scotland, and the simple features held great in all weather with a light load (6-7kg).

  \item [Osprey] Over the years, Osprey has made a name for themselves by making very decently priced, sturdy and not too heavy packs. I’ve own an Exos for a few years, and still take it for a spin every once in a while. This was the pack I brought with me in the Himalayas, and I couldn’t have been happier with its performance with a medium load (9kg).
  \item [Sierra Design] A very interesting brand if only for the fact that they don’t do 100\% symmet- rical packs ! Think about how often you strap the same thing on the same side of the pack ? Well, maybe having a symmetrical design is not the best option. I certainly like the concept.

  \item [ÜLA] ÜLA is definitely a good place to look at. It’s a refreshing combination of ultralight - very simple - designs, with a more complex cut (not the usual rectangle pack) then most.

  \item [MLD] Mountain Laurel Design is also one of my personal favourite, although I never had a chance to try one out. They work from basic shapes, but have tons a very smart details that make it a great selection of packs.

  \item [ZPacks] Their packs are simple but effective. And they have a frame ! Which is quite rare for UL designs. I haven’t had a chance to wear one yet and see how it carries.

  \item [Decathlon] A big outdoor name from France who does more research on fabrics and designs than people might think. They do lack specialised equipment, but instead try to innovate a lot, so I always keep an eye on what they come up with. I’ve own many of their day packs, and have gotten my hand on many different trekking/hiking designs, and they all came up surprisingly close to what I consider lightweight.

\end{description}
