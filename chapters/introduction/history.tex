The outdoors always helped me balance my somewhat stressful and chaotic work life with some blessed moments of calm and quiet. Before taking on sewing, I had spent a lot of time hiking around the world, and although I don't consider myself an ultralight backpacker, I do make conscious decisions on what to pack on a trip based on weight and versatility. At the end of the day, it always came down to my gear not being one hundred percent suited to my backpacking style, to the gear I usually carried, or simply to the aesthetics I was attracted to. More often than not, I ended up with having to carry more than necessary, or spent more money than I should have on more specialised gear.

Designing my own backpacks came quite naturally. I happened to already own a handful of them, each one with their pros and cons. I knew what I liked or disliked, what I used, and what I did not. I've always slightly modified my packs to remove unnecessary weight on them, just so I can focus on carrying the essential on a longer distance, or simply with less efforts. All that trimming brought up many questions - and answers - about the structure of today's backpack, their versatility, and - by design - their average target group.

From there on, the jump to making a complete pack was mostly about acquiring the skills, researching the fabrics and components, and gathering tons of drawings in the process. It is from those drawings that I started putting together my first packs. Suffice to say, my initial attempts barely qualified as functional, but they did serve a purpose. I used the flaws of each pack to enhance the next design, and the next, and so on. I also started paying more attention to all the packs I could get my eyes and hands on. I spent time analysing those designs and foraging through other manufacturer's collections just so I could build up some kind of a knowledge base on how other people solved the same problems I was facing (or was about to face). To complete my training, I got my hands on a sewing machine and started practicing as often as I could, sewing together pouches, stuff sacks, and small yet more complicated items using, for example, velcro or zippers. The idea was to slowly ramp up towards putting packs together, which was my goal from the start. The next logical step was to use different - more specialised - fabrics, and experiment with less ordinary components, until I finally built up the confidence to work on more complex pack designs. I will present one of my most recent packs throughout this document to illustrate my words with a real life example.

You will have noticed it by now, I am the kind of person who can spend hours thinking about how to do something before ever getting hands-on. Now, designing a backpack has a lot to do with planning ahead, but - as I learnt since - there is no amount of planning which can compensate for the lack of getting your hands dirty. Anyone willing to embark on this adventure should consider balancing equally designing and planning time with time for experimenting and building mock-ups. There is - in the end - no time better spent on than putting together an inexpensive and simplified mock-up only for the purpose dissecting it thoroughly. Keep this in mind before spending money on more expensive fabrics and potentially making a mess of it. On the plus side, if one of these mock-ups turns out alright, you can always offer it to friends and family.
