
The outdoors always helped me balance my somewhat stressful and chaotic work life with some blessed moments of calm and quiet. Before taking on sewing, I had spent a lot of time hiking around the world, and although I don't consider myself an ultralight backpacker, I do make conscious decisions on what to pack on a trip based on weight and versatility. At the end of the day, it always came down to my gear not being one hundred percent suited to my personal needs, and more often than not, ended up with me having to carry more than necessary.

Designing my own backpacks came quite naturally. I happened to own already a handful of packs already, each one with their pros and cons. I knew what I liked or disliked, what I used, and what I did not. I've always slightly modified my packs to remove unnecessary weight on them, just so I can focus on carrying the essential on a longer distance, or simply with less efforts. All that trimming brought up many questions - and answers - about the structure of today's backpack, their versatility, and by definition their average target group.

From there, the jump to making a complete pack was mostly about acquiring the skills, researching the fabrics and components, and gather tons of drawings. It's from those drawing that I stated putting together my first pack. Suffice to say, first attempts barely registered as functional, but I used the failures of each to improve the next pack, and the next, and the next. I started paying more and more attention to all the packs I can get my eyes and hands on, analysing more and more designs to build up some kind of a knowledge base on how others solved existing problems. On the other hand, I got started training, and kept on sewing together different kind of packs, using a multitude of fabrics building more and more complexity in each pack.
