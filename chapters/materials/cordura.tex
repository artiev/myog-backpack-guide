Cordura is a fantastic fabrics to work with, especially coated. Uncoated, it's a rather soft fabric (depending on the denier it comes in) and frays very easily. One thing you should consider is that coated Cordura is much stiffer than uncoated, resulting in you probably needing a lower denier then uncoated. Denier is the rating of the thread of fibre used to weave the fabric, the higher the denier, the heavier and more durable the fabric.

As it is mostly comprised of nylon fibres (normally solely, but it sometimes comes with a cotton blend), it does not soak up water or moisture much, meaning its weight when wet is not significantly higher than when dry.

Cordura is a great fabric for parts of a backpack which will have to endure more abrasion and stress than the rest. This is usually true of the bottom part of a pack, when it is standing on the ground, and when items inside apply pressure at certain points.


\begin{figure}[H]
  \includegraphics[width=\textwidth]{media/images/fabrics-cordura-set-1}
  \caption{Cordura sample set}
  \label{img:fabrics-cordura-set-1}
\end{figure}


\begin{figure}[H]
  \includegraphics[width=\textwidth]{media/images/fabrics-cordura-set-2}
  \caption{Close-up of different denier Cordura weaves}
  \label{img:fabrics-cordura-set-2}
\end{figure}
