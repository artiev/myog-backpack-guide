I have spent quite a few bucks on different kinds of fabrics. There has been a few which really stood out from the rest by how strong they feel, or how easy they are to work with. There is a few things I'm looking for in the fabrics I use apart from the aesthetics:

\begin{description}

  \index{ripstop}
  \item [Ripstop] A backpack will get punctured, scratched, tossed around, overloaded and so on. If there is one thing I expect from the fabric, is that it survives for the duration of my trip no matter what tear and wear the pack goes through. Ripstop treatment is the process of interweaving a stronger filament at short intervals within the fabric, with a recurring pattern - say 5 mm by 5 mm squares - so that a tear in the fabric will be stopped by the stronger thread. It also helps the sewing process by offering anchors, and makes a fabric overall much more durable.

  \index{coating}
  \index{waterproof}
  \item [Water resistant coating] There is one thing that does not work well for traveling light, and it's having a fabric that absorbs water, making the pack much heavier than it should be. There is plenty of methods for coating a fabric, and there is enough demand for waterproof fabrics that the price of coated versus uncoated is close enough not to even care.

  \index{humidity}
  \item [Soaking resistance] Different fabrics come with different blends of material. Some of these materials' fibres resist much better to water than others. When I am considering a fabric, the water resistant coating is one thing, but it's ability to not get heavier with rain or humidity is almost as important. As the fabric does not weight more, and does not need drying.

\end{description}
