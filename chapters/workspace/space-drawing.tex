I find most of my inspiration on the web. There are people out there who have better ideas than mine, and I don’t shy away from blatantly experimenting, learning and copying any concept or design I like. I am dedicating the entire chapter \ref{chap:inspiration} to how and where I find inspiration, so let’s just boil it down to the essentials here and look at the image \ref{img:workspace-2}.

\illustrate{media/images/workspace-2}
{My desk for researching and drawing new packs}
{img:workspace-2}

\begin{description}

  \item[Paper] I prefer a plain white fine grain A3 Canson sketchbook (I just love the space and the grain, and the lack of constrains of plain paper) but any paper/sketchbook would do of course. I stay away from millimetre paper because I don’t need accurate drawings.

  \item [Ball Pen] Just trust me, and don’t use a pencil and an eraser. It’s way better to draw, make mistakes, and draw again and again. Keep your complete history of designs. Take notes on them regarding why you would have erased it, what you think works and doesn’t.

  \item [Cutting Mat] A cutting mat is awesome to keep around as it gives a scale to anything you find online.

  \item [Measuring Tape] A tailor’s tape measure, so I can grab any of my packs and take a fresh set of measures to scribble it into a drawing. Any ruler would do, but working with fabrics and a rigid ruler is not ideal.

  \item [References] I keep some kind of inventory of the components I’ve tried and liked. I have tried loads of different webbing types and sizes, different buckles, different clips. Keeping track of whether they work together or not is essential. This is great to already get an idea of how I can put something together. See section \ref{sec:gear-database} for more details.

\end{description}
