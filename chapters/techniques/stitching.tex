I find it cumbersome to use a heavy-duty thread when sewing together a backpack. Parts of the pack which involve stitching together multiple layers of fabrics can sometime be quite tough on my sewing machine if the needle combined with the thread make for a wide diameter to punch through. So I use a thinner thread instead (see \ref{sec:sewing-thread}).

\index{triple stretch stitch}
\index{backstitch}
In combination with this, I use a special stitch, often referred to as the triple stretch stitch or backstitch, which is very robust yet flexible but still uses a single thread and a single needle. The result is much stronger than the tensile strength of the thread itself, and its construction has the added benefit of rendering a single point of failure almost impossible. As the triple pass (two times forward, one time backward each step) prevents any snapped thread from getting loose, resulting in a catastrophic failure of your seam.

The backstitch also allows a seam to stretch beyond what the thread's would normally offer, resulting in overall less stress on a single thread under heavy loads.
