\subsection{Account for the seams before you cut}\label{sec:seams-and-cutouts}

In most of the drawings in this documentI only highlight the basic dimensions of the finished panels and add a symbolic outer seam allowance.

Before cutting any panels for a backpack, you should pay attention to all the seams and extra fabric necessary to bind and build all the features of each panel.

The illustration \ref{ske:panel-back-seam-highlight} is an example of how seams can impact fabric cut-out of the panel shown in illustration \ref{ske:panel-back-design}, and why you should always have a few drawing of the features of your pack. In this instance, I would cut 4 separate panels and assemble them into a complete back panel before starting working on another panel.

\begin{note}
  When working on design which involve multiple parts per panel, you also want to consider the fabric's patterns, and make sure they match!
\end{note}

\index{seam allowance}
Seam allowance is crucial to the construction of a backpack. After experimenting a lot, I realised I tend to sew seams 10mm to 15mm from the side of the fabric. With that in mind, if I leave more than 15mm of fabric for the seem allowance - say 30mm - I would make a mess of my well thought pack dimensions because every seam would add overall 30mm extra fabric for each bond. That will end up in a much bigger pack than originally designed.

\illustrate[.75]{media/sketches/panel-back-seam-highlight.pdf}
{Be careful to always consider your dimensions with and without the extra seams}
{ske:panel-back-seam-highlight}

\begin{note}
  It's also the perfect allowance for binding the seams at a later stage with 20mm grosgrain ribbon.
\end{note}

\subsection{Side pockets and compression straps working together}\label{sec:construction-side-pockets}

\subsection{Bottom section cut}

\illustrate{media/sketches/pack-bottom-section-cut.pdf}
{Section cut of the bottom construction including reinforcements and padding}
{img:pack-bottom-section-cut}
